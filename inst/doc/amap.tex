% building this document: (in R) Sweave ("ctc.Rnw")
\documentclass[a4paper]{article}

\title{Amap Package}
\author{Antoine Lucas}
%\VignetteIndexEntry{Introduction to amap}
%\VignettePackage{amap}

%\usepackage{a4wide}

\usepackage{/usr/local/R/R2.2.1/lib64/R/share/texmf/Sweave}
\begin{document}

\maketitle

\tableofcontents

\section{Overview}

{\tt Amap} package includes standard hierarchical  
clustering and k-means. We optimize implementation 
(with a parallelized hierarchical clustering) and
allow the possibility of using different distances like
Eulidean or Spearman (rank-based metric).

We implement a principal component analysis (with robusts methods).

\section{Usage}

\subsection{Clustering}

The standard way of building  a hierarchical clustering:
\begin{Schunk}
\begin{Sinput}
> library(amap)
\end{Sinput}
\begin{Soutput}
Le chargement a nécessité le package : Biobase

Welcome to Bioconductor 

	Vignettes contain introductory material.
	To view, simply type 'openVignette()' or start with 'help(Biobase)'. 
	For details on reading vignettes, see the openVignette help page.
\end{Soutput}
\begin{Sinput}
> data(USArrests)
> h = hcluster(USArrests)
> plot(h)
\end{Sinput}
\end{Schunk}
 Or for the ``heatmap'':
\begin{Schunk}
\begin{Sinput}
> heatmap(as.matrix(USArrests), hclustfun = hcluster, distfun = function(u) {
+     u
+ })
\end{Sinput}
\end{Schunk}
On a multiprocessor computer:
\begin{Schunk}
\begin{Sinput}
> h = hclusterpar(USArrests, nbproc = 4)
\end{Sinput}
\end{Schunk}
The K-means clustering:
\begin{Schunk}
\begin{Sinput}
> Kmeans(USArrests, centers = 3, method = "correlation")
\end{Sinput}
\end{Schunk}

\subsection{Robust tools}

A robust variance computation:
\begin{Schunk}
\begin{Sinput}
> data(lubisch)
> lubisch <- lubisch[, -c(1, 8)]
> varrob(scale(lubisch), h = 1)
\end{Sinput}
\end{Schunk}
A robust principal component analysis:
\begin{Schunk}
\begin{Sinput}
> p <- acpgen(lubisch, h1 = 1, h2 = 1/sqrt(2))
> plot(p)
\end{Sinput}
\end{Schunk}
Another robust pca:
\begin{Schunk}
\begin{Sinput}
> p <- acprob(lubisch, h = 4)
> plot(p)
\end{Sinput}
\end{Schunk}


\section{See Also}

Theses examples can be tested with command
{\tt demo(amap)}.\\

\noindent
All functions has got man pages, try 
{\tt help.start()}.\\

\noindent
Robust tools has been published: \cite{caussinu+ruiz} and
\cite{caussinu+ruiz2}.


\bibliographystyle{plain}
\bibliography{amap}



\end{document}


